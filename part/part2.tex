\chapter{Le lien éducatif}

\emph{Je suis français, je suis Espagnol, Anglais, Danois. Je ne suis pas un, mais plusieurs. Je suis comme l'Europe, je suis tout ça.} L'Europe agit dans l'enseignement supérieur mais moins dans d'autre domaine. Pourquoi ce rôle d'éducation est-il encore limité, pourquoi y a-t-il des résistances dans les États.

Doubles enjeux de l'éducation: 
\begin{itemize}
\item Enjeux politiques, culturels et sociaux
\item Enjeux économiques
\end{itemize}

\textbf{Enjeux culturels politiques et sociaux relatifs à l'éducation}

Emile Durkheim (1880), sociologue qui s'intéresse à l'éducation. Il souligne le rôle primordial de l'éducation dans l'éducation dans la formation des citoyens. L'éducation va être un facteur de cohésion sociale. A travers l'éducation, il y a transmission de normes et de valeurs. Homogénéité entre les individus qui leur permettent de vivre ensemble.

Assoir le régime républicain contre le régime monarchique. Par l'intermédiaire de l'école, instauration d'une morale civique. L'école a une dimension politique très forte.

État moderne de l'éducation, dont le rôle est de garantir l'unification et la reproduction culturelle de la société.

Pierre Bourdieu comme outil de reproduction des élites et de l'ordre social.
\begin{itemize}
\item Société à mode de reproduction scolaire\\
Les status des individus dans la société sont en partie légitimé par le diplôme que l'on possède. On n’hérite pas de sa situation mais on l'obtient grâce à son diplôme.
\item Dénoncer le mythe méritocratique\\
La réussite scolaire est lié très largement à l'origine sociale, c'est ce que dit Bourdieu. Notion de capital culturel.
\end{itemize}



\textbf{Enjeux économiques relatifs à l'éducation}

L'éducation est une façon de produire des richesses. Gary Becker, économiste.
\emph{L'éducation augmente la productivité des individus. Notion de capital humain.}

Attention: capital humain différent du capital culturel de Bourdieu.


\begin{itemize}
\item Investissement\\
dont le rendement dépend du rapport entre le coût des études et le gain anticipé sur le marché du travail.
\item calcul coût / intérêt.
\item Bénéfice aussi au niveau du pays dans son ensemble (externalité positive).\\
Retombé sur la société (progrès scientifique, innovation, expertise, création, culture).
\end{itemize}

Liens avec le taux de croissance: économie de la connaissance.\\

Pour étudier les politiques publiques
\begin{itemize}
\item construction et production au niveau des instances dirigeantes
\item mise en place de ces politiques
\item Effets de ses politiques\\
A l'échelle de l'individu, d'une université, etc...
\end{itemize}


\textbf{Problématiques}

Comment les échelles prises à l'échelle européenne au niveau des instances de l'Union européenne ou de négociations inter-Etats se répercutent-elles dans les différents pays concernés? 

assiste-t-on à une convergence des systèmes nationaux et à la création d'un modèle européen d'enseignement supérieur? 

Enfin, à l'échelle des acteurs de l'enseignement supérieur (acteurs institutionnels ou individuels) quels sont les effets de cette mesure?

\textbf{Processus d'européanisation:} en tant que domaine de la vie sociale et de l'action publique, comment l'enseignement supérieur est-il transformé par la construction européenne, et plus généralement \emph{européanisé}? (effets indirects de la construction européenne)

\textbf{Processus d'intégration:} alors que la gouvernance de l'enseignement supérieur a pendant longtemps relevé exclusivement du niveau national, assiste-t-on à l'émergence d'un niveau européen de gouvernance? (création directe par l'UE d'instance de gouvernance européennes)

\section{Quelle politique européenne pour l'enseignement supérieur?}

\subsection{Quelles politiques européennes pour l'enseignement supérieur?}

	\subsubsection{Un questione tardive}

\begin{itemize}
\item Traité de Rome de 1957\\
(texte fondamental de la CEE): aucune référence (sauf enseignement pro).
\item Traité de Bruxelles de 1965\\ 
C'est la création de la Commission et du Conseil des ministres. Mais à nouveau, rien en rapport avec l'éducation.
\item Seulement un projet notable: 1976, Institut Universitaire Européen de Florence (IUE).  
\end{itemize}

Harmonisation ou coopération? 

- État soucieux de garder leur souveraineté dans le domaine éducatif.
Première réunion des ministres de l'éducation en 1971.
Cette idée d'harmonisation suscite de nombreuse réactions négatives. La commission remplace ce terme de coopération.

L'Union Européenne n'intervient que lorsque l'État seul est impuissant.

Une coopération réduite: 
\begin{itemize}
\item 1976 (comité de l'éducation. Aucune compétence décisionnelle ou législative).
\item 1980: réseau de recherche et d'information Eurydice.
\end{itemize}

Mission de ce réseau se précisent au fil du temps:
\begin{itemize}
\item Echange d'information, production ou documentation
\item Production d'outils de comparaisons et d'indicateurs\\
1990: faciliter l'élaboration des analyses comparatives
\end{itemize}

--> Jusque dans les années 1980: pas de programme commun. Système d'échange d'information, pas encore très structuré.
	
	\subsubsection{La mise en place d'Erasmus}

C'est le programme le plus ancien. Programme emblématique, doubles enjeux:
\begin{itemize}
\item favoriser la construction d'un esprit européen
\item former une main-d'œuvre qualifié et internationale\\
Lancement du marché unique 1986.
\end{itemize}

Même s’il y a une recherche de coexistence de deux logiques, il y a une résistance des États membres à voir la Commission européenne jouer un rôle.

La Commission est légitime à intervenir dans le domaine de la formation professionnelle. Avant 1985, il n'y a aucun texte de la CEE qui autorise à prendre des décisions, des mesures à l'enseignement supérieur. Un arrêté a permis de lancer le programme Erasmus de façon à ce que l'éducation à l'étranger soit facilement accessible et égalitaire par rapport aux étudiants vivant là-bas.

Programme Erasmsus, 3 actions:
\begin{itemize}
\item Développement d'un réseau de coopération
\item Appuie financier à la mobilité grâce à un système de bourses
\item Amélioration de la reconnaissance académique
\end{itemize}

Le programme Erasmus est rejeté 3 fois avant de l'accepter en juin 1987. Ce n'est pas une obligation, dépend des initiatives locales.

Traité de Maastricht 1992 qui réaffirme la subsidiarité.

1995: Programme Socrate 1, programme de coopération transnationale dans le domaine de l'éducation.

Idée: échange information + favoriser la mobilité et un apprentissage des langues.

	\subsubsection{Le processus de Bologne}
	
	\subsubsection{La stratégie de Lisbonne}






\section{Les effets de la construction européenne sur les systèmes nationaux d'enseignement supérieur}

\section{Européanisation ou mondialisation de l'enseignement supérieur?}

