\chapter{Le lien marchant, l'économie}

Communauté européenne du charbon et de l'acier, tenu 51 ans arrêté en 2002. Cette place c'est élargie au point de devenir un évènement fondateur de la solidarité européenne. Marché unique avec monnaie unique et qui engage des politiques [...].

\section{Idéologie et performabilité}

Quels sont les idéaux qui ont lancé la formation de l'Europe?
Performativité: linguistique, Austin: \emph{in distingue deux types d'actes de langage. Les constatifs et les performatifs (ceux qui produisent un monde).} On reste sur la performativité d'un discourt. Les idéologies qui ont été défendu ont données lieu à la construction d'un modèle. Le monde tel qu'il se construit se forme à l'image des idéaux qui l'on construit. Comment ces pensées initiales ont performé l'Europe économique? 

C'est une crise concentrique qui va de la banque -> finance -> économie -> social. Les sphères économiques et sociales prennent du temps pour être impacté. L'Europe n'a pas pu faire face à cette crise économique. Les journalistes cibles ces phénomènes de dérégulations. Ces états ne parviennent plus à contrôler le marché.

L'impuissance de l'Europe politique peut être expliqué à partir de son origine, juste après la seconde guerre mondiale. Prendre un dogme, l'économie dogmatique, au départ ce n'est pas une réalité, cela n'en reste pas moins une idée.

\subsection{remonter dans le temps pour comprendre l'arrière-plan des crises actuelles}

Pourquoi avoir fait du marché, le socle ? À quel modèle économique le marché unique européen va-t-il s'adosser? Quel sera le modèle économique européen? 

Après seconde guerre mondiale: tensions en europe. Comment allons-nous faire pour nous entendre? Deux modèles: idéalisme fédéral (USA) et de l'autre réalisme intergouvernemental.

\subsubsection{La communauté européenne pour le charbon et l'acier (CECA), pour la volonté de s'associer avec l'Allemagne, pour une pacification.}

La déclaration de Georges Schuman amène à ce désir de paix.
La CECA (signé en avril 1995).
L'Europe dans ses bases même étaient très largement libérale. 

\subsubsection{La communauté européenne de défence}

Lutte contre le communisme. La politique ne servira pas de glue à l'union européenne. Les passions ne permettent pas de faire vivre le collectif. L'intérêt, l'économie est une tentative de mise en place de cette sécutiré. Pas de solidarité naturelle entre nous mais si nous nous mettons à échanger, des liens vont sans doute se créer. C'est par l'économie que l'on va aller à une entente plus forte.

Intégration négative plus importante que positive. Volonté de souscrire à l'économie libérale. Le marché et l'économie est un moyen non un objectif, l'objectif est l'union des différents états européens.

Le marché a servi à dépassionner en injectant de l'intérêt. La recherche du profit est une tension calme. Aujourd'hui, le marché ayant l'intérêt comme dogme principe à du mal à répondre à cet initial objectif.


\subsection{A quel modèle économique adosser le marché unique européen?}

Le totalitarisme socialiste. Des modèles qui privilégient la liberté.
- En même tant, il y a la monté du néolibéralisme. Lippman suggère que l'intervention de l'état doit intervenir sans s'opposer à l'économie. Le marché doit être une sphère auto régulée.

- l'ordo-libéralisme (une partie du néolibéralisme) allemand. Montrer comment ces modèles économiques jouent sur le marché. Les territoires économiques sont animés par des idéologies, des modèles par des acteurs. Qui sont ces acteurs? Ce sont les politiques qui nous gouvernent? Les hommes politiques n'ont plus cette compétence. Les hommes politiques ne sont plus expert. Seul les économistes sont expert. Certains économistes ont une influence sur les politiques donc les décisions sont orienté par ces idées de certain économiste.

En 1957, c'est l'ordo libéralisme qui va être appliqué car il croise politique et économie (oximore entre ordre et libéralisme). Un ordre politique qui est là et qui préserve l'économie libre.

La loi de 1905 permet d'assurer au marché un fonctionnement optimal. Ordo-libéralisme, associé au miracle économique Allemand. Il se positionne contre les illusions nationalistes. Il faut laisser le marché se développer avec ces propres logiques.

Cette idéologie va être renforcé par les crises des années 70 et l'arrivée au pouvoir de quelque penseur de l'économie. L'importance de la crise 73, la signature de l'acte unique le 28 février 1986 est une étape vers un marché toujours plus libre, le traité de Maastricht.

\subsection{insuffisance du lien social}

- Conséquences négatives sur la solidarité sociale
- Retrait des politiques publique
- Recul des politiques macro-économiques
- Recul de l'Etat providence










\section{Territoire et encastrement}

Comment l'Europe c'est construite en tant que territoire marchand? Europe: l'idée est de supprimer les frontières à l'intérieur. L'autre idée est de durcir la frontière autour de l'Europe (entre interne et externe à l'Europe).

Règles, cultures, liens sociaux, le nouveau espace européen doit composer avec un social déjà fondé. L'encastrement culturel échape au juridique, le culturel dépend de beaucoup (local, éducation, ...). 

Montrer à la fois une construction et politique commune mais nous allons voir qu'il y a aussi des résistances à ces nouvelles constructions par rapport aux cultures.

\section{Dynamique et interdépendance}

Interdépendance à travers les divers marchés européens. Réfléchir sur la place de l'Europe pour limiter les débordements de cette crise. Crise de dérégulation. Evolution des rapports de force entre les politiques et les marchés.
