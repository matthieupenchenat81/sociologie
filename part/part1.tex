\chapter{Le liens politique}

On s'interroge sur ce qu'est l'Union Européenne, c'est un objet inédit. Jacques Delors, président de comité européenne a dit que c'était un opti (objet politique non identifié). On a jamais rien vu de comparable. Elle est la transformation politique la plus importante des 60 dernières années, c'est la première fois que nous avons une période si grande sans guerre entre les pays européens. C'est inouï.

C'est un objet qui change tout le temps, la Grande-Bretagne envisage de partir. C'est une communauté politique à territoire variable. Il y a des fonctions variables. Cette dynamique est en partie contrainte. Cette dynamique est en partie contrainte. L'Europe n'est pas seulement un projet positif, le projet fondateur est d'arrêter les guerres entre eux. 

\section{Le lien européen, le lien contraint. Les origines de l'Europe}

La construction européenne est à la fois un projet politique fort et c'est la volonté d'en finir avec les guerres entre les pays européens et à la fois c'est les fruits des contraintes d'une histoire. ET à même temps, cela répond à des contraintes qui arrive à des exigences américaines.

\subsection{L'exigence américaine} 

L'Europe actuelle, l'Europe contemporaine comment le 19 septembre 1946, Windston Churchill, premier ministre de la Grande-Bretagne à l'époque propose la création des états-unis d'Europe qui d'après lui serait le troisième pilier du monde occidental.

Quel serait les deux premiers? 
-	Le Commonwealth empire Britannique.

Les Britanniques pensent qu'ils sont au croisement du monde occidental c'est d'être tri polaire avec les USA, la zone d'influence Britannique. Churchill met à l'agenda la construction européenne, faut-il le faire et si oui comment? 

L'élément le plus important, c'est le plan Marshall. Les USA se déclarent prêt à aider massivement l'Europe aussi bien en terme financier qu'en terme technologique et à leur transférer des technologies aussi.
Mais ils y mettent une condition: la gestion de l'aide américaine doit être faite sur une base européenne.  Les USA considère les européens comme immature politiquement. Ce qui veulent les américains c'est que les européens s'organisent entre eux pour construire une organisation européen.

Le troisième point important, le coup de Prague:
Les Russes interviennent en Tchécoslovaquie, oblige les pays européens à se poser la question du danger russe. En 1948, l'Allemagne vient d'être vaincu, est-ce prudent de laisser l'Allemagne complètement désarmée? Entre eux et la Russie.
Pour toutes ces raisons, ne serait-il pas tant que l'Europe s'organise face à l'Amérique? Ne faudrait-il pas construire l'Europe à cause du souhait américain et à cause de l'oppression soviétique. Les USA sont obligés de coopérer.

Deux ensembles, deux options: 
-	favorable à une union (les fédéralistes): Belgique, Hollande, Luxembourg, France.
- 	Les intergouvernementalistes: la coopération la plus réduite possible. L'Angleterre et les pays Nordiques. Coopération la plus réduite possible.

Opposition presque inconciliable entre deux visions de l'Europe. Cela donne des débats dans les années 40 sur la gestion de l'aide Américaine. Qui dois gérer l'aide Américaine? Les fédéralistes souhaite que l'aide USA soit gérer par une institution puissante et puis les intergouvernementalistes veulent que l'aide Américaine soit simplement gérer par un simple secrétariat qui est le mandat le plus réduit possible. Ce sont les fondamentalistes qui gagnent qui fondent l'OECE (organisation européenne de coopération économique) qui deviendra quelques années plus tard l'OCDE (organisation coopération et développement économique) qui n'a pas beaucoup de pouvoir, pas d'élection, pas de contrôle et c'est simplement bureaucratique.

Deuxième débat dans les années 40 sur comment organiser l'européenne, fondamentaliste avec des élections européennes avec des politiques européenne, le conseil de l'Europe, les intergouvernementalistes n'en veulent pas mais il est tout de même formé (avec tout de même mandat réduit sur le fait de contrôler les droits de l'homme). Les intergouvernementalistes vont à nous réussir à imposer leur vision.


Une exigence économique et industrielle


Les fédéralistes ne sont pas contents, aucune organisation ne leur convient et donc le 09 mai 1950, Robert Schuman qui est ministre des affaires étrangère de la France de l'époque propose une idée qui va être très importante. Il propose à l'Allemagne et à tout les pays intéressé, en mettant en commun toute la production du charbon et de l'acier. L'idée est de rapprocher l'Allemagne et la France, interdépendant. EN mettant en commun ce qui à l'époque était les deux industries les plus importantes.
C'est l'idée qui va être accepté l'Europe des 6 (Allemagne, Belgique, Pays Bas, Luxembourg, Belgique et la France).

Ces 6 pays fonde une communauté européenne du charbon et de l'acier et il y a deux façons de voir cette tentative. De façon positive. C'est concret, modeste mais qui marche. Et négativement on a vraiment renoncé aux grandes ambitions. Plus du tout l'objectif de refonder l'Europe. Robert Schuman revendique la modestie et le caractère concret car il pense que les pays européens ne pourront avoir la même vision de l'Europe.

Il propose de renoncer aux ambitions politiques fortes et de se concentrer sur des coopérations concrètes mais réelles. Ce qu'il propose c'est de contourner la politique par la technique, constitution plus simple et coopération concrète et précise.

Le 25 mars 1957, les traités de Rome.

Au début des années 80, tout le monde se rend compte au début des années 80, que la crise économique qui a commencé des années 75 est durable et le contexte change. Ils changent aussi car les hommes politiques changent.

------


On peut mentionner que le royaume-uni, le Danemark et la Suède ont refusé l'adoption de l'euro. En 2005, la France a refusé par un référendum. Le 29 mai 2005, la France a refusé par un référendum.

La Suisse n'est pas membre de l'Union Européenne. La Turquie veut être membre et la Suisse ne veux pas. Des pays comme la Grande-Bretagne sont toujours extrêmement réticent vis-à-vis de l'euro.

\section{la construction d'une apparente irréversibilité}

Les explications théoriques par rapport à cette histoire.

L'Europe c'est le grand sujet politique depuis 50 ans. Opposition entre ceux qui pensent que l'UE est un jeu à somme nulle (on gagne tout ou on perd tout).

Et puis on la considère de plus en plus comme un jeu où on gagne de plus en plus, plus on joue, plus on gagne.

\subsection{TODO}

Longtemps la question c'était de savoir qui gagnait à la construction européenne et qui gagnait. Entre les intergouvernementalistes et les néofonctionnalistes.

- Les intergouvernementalistes (les gouvernements sont les grands gagnants de la construction européenne). Alan Milward, qui pense établir essentiellement trois choses:
\begin{enumerate}
\item Les gouvernements continuent à prendre les décisions à Bruxelles.
\item La construction européenne à avantagé les gouvernements puisque elle a déplacé les sujets les plus importants à Bruxelles.
\item La coopération européenne leur sert de bouc émissaire. Toutes les décisions impopulaires sont attribués au gouvernement.
\end{enumerate}

- Les néofonctionnalistes (Herms Haas), c'est le niveau européen qui s'impose. C'est l'engrenage. C'est ce que contrairement à ce que pense les intergouvernalistes, par effet d'engrenage, au début, les pays européens ont coopéré sur des sujets peu politiques et puis par effet d'engrenage, ils ont été entrainé malgré eux dans un engrenage parce qu'on est dans des sociétés pluralistes, dans des sociétés démocratiques. Il y a des débats, des oppositions d'intérêt. Et quand un groupe perd la bataille pour imposer ses vues, il cherche des alliés, il renforce ses alliés. Un allié possible est l'allié européen. Il y a une possibilité pour tous les acteurs nationaux, la possibilité de s'allier avec les acteurs européens. Les acteurs nationaux ne gagnent pas durablement.

La question importante est comment arrête-t-on les guerres? 
Bien sur, il y a à la fois une opposition totale entre l'internationalisme et le [...]
Complémentarité politique. L'analyse intergouvernementaliste correspond bien avec certaines politiques. Elles permettent bien de comprendre les phases critiques et les moments de ruptures.

L'analyse néofonctionnaliste est bien adapté à d'autre domaine. En même temps, elles expliquent des analyses.

\subsection{La thèse du jeu coopératif}

Tous ceux qui coopèrent gagne. Keohane et Nye ont écrit un livre, "pouvoir et interdépendance", c'est l'idée que l'on a des problèmes de plus en plus globaux qui ne peuvent plus être résolu seulement au niveau national (pollution, chômage).

L'idée de base est celui de gouvernance. Ce terme apparait au milieu des années 70. Il est extrêmement critiqué, débattu, polysémique mais il désigne fondamentalement un mode de coopération ou de décision qui ne repose pas sou pas seulement sur la hiérarchie. Il y a gouvernance lorsque nous ne savons pas exactement qui décide.
Quand il y a un décideur, ce n'est pas une gouvernance. Gouvernance = personne ne peut décider tout seul.

La gouvernance apparaît au milieu des années 1970. Mai 1968, l'ingouvernabilité. Le gouvernement c'est fini. Les gens ne sont plus toujours d'accord. C'est la contestation sociale c'est aussi l'émergence du tiers monde. Les pays nouveaux qui ne sont pas d'accord. Un certain nombre d'acteurs vont raisonner autrement. Il faut inventer autre chose que le gouvernement, la gouvernance.

Rock Feller, un millionnaire américain est un premier penseur qui remarque que les contestations sont un risque pour le capitalisme (Warning, pas sur). Il a extrêmement peur d'une remise en cause du capitalisme. Il fonde la commission trilatérale en 1973. Elle est très originale pour deux raisons, qui regroupe des industriels, des politiques et des industriels.

Il pense qu'il fait réunir les industriels, les hommes politiques pour imaginer un monde nouveau. Tout les deux trois ans, elle confie un rapport à des universitaires pour réfléchir à comment organiser le monde. EN 1973, le premier rapport est publié, "la crise de la démocratie", publié par Samuel Huntington. Le deuxième auteur est Michel Chrosier et le troisième est Watanuki, extrêmement important qui dont la thèse des modèles des gouvernements n'est plus adapté à la situation actuelle car les administrations ont intérêts à se développer, à recruter de plus en plus de fonctionnaire, à avoir de plus en plus de mission. Il y a donc une surcharge, ils sont de plus en plus lourd et font les choses de moins en moins bien. Une première visions de la gouvernance qui est celle d'un affaiblissement de l'État, d'un allègement de l'État. Vers un État qui se recentre vers ses missions fondamentales.

La seconde phase de la gouvernance est une phase beaucoup plus gauchiste, elle réclame la participation du partenariat. Il y a une forme de contradiction entre les deux formes de la gouvernances.

La gouvernance européenne, le terme c'est beaucoup développé puisque l'Europe s'est développé, négociation se conclu sans [...]. Le terme le plus développé est la gouvernance multi-niveau. C'est la thèse qui est le plus en plus développé. Leibfried et Pierson ce sont les principaux inventeurs de gouvernance à propos des politiques sociales avec l'idée que la négociation est de plus en plus double. Elle porte d'une part sur les politiques et les réformes à adopter et d'autre part sur le niveau que doit prendre la décision. L'Europe, l'État ou la région. Gouvernance multi-niveau négociation au niveau politique et où elle doit être prise. C'est quand vous considérer qu'on ne sait pas du tout où est le gagnant.

\subsection{Les dynamiques régionale qui contribuent a l'européanisation}

Depuis une vingtaine d'années, toute une série d'auteurs ont essayé d'analyser. Avec l'idée qu'effectivement les gouvernements ne sont plus seul. Les acteurs locaux et régionaux pèse de plus en plus dans la négociation et que finalement les acteurs régionaux et locaux sont souvent partisans de la construction Européenne parce qu'elle les libère de la tutelle de l'Etat.

À partir de la, on va se fier à un auteur Romain Pasquier, un Breton, il a écrit un livre, "les capacités politiques des régions" en 2004. Il compare deux régions, la Bretagne et le Centre et deux régions espagnoles: la Galice et la Riogia. 2 régions française et espagnole. EN France comme en Espagne, on a des régions de plus en plus puissante. En France, la décentralisation de 1982 a libéré les régions de la tutelle des préfets mais les régions restent relativement faibles, elles n'ont pas été redécoupé. On n’avait pas modifié la fiscalité. Les régions ne représentent que a peu près 3\% des dépenses publiques. Les régions ne se sont pas beaucoup servit de l'Europe. Finalement les programmes Européens ont donné du pouvoir aux préfectures des régions.

En Espagne, les pouvoirs des régions sont bien plus étendu et en plus ils sont toujours croissant parce qu'en Espagne, il y a une course entre les régions pour avoir toujours plus de pouvoir; avec des régions qui considèrent qui sont différentes (la Catalogne et le Pays-Basque). En Espagne, il y a une forme d'autonomie croissante du pouvoir central. En Espagne, les programmes Européens ont renforcé les régions. 

Globalement, les régions espagnoles ont été renforcé par l'européanisation alors qu'en France, ce sont plutôt les fonctionnaires qui ont été renforcé. Finalement, c'est beaucoup plus compliqué lorsque l'on arrive à la comparaison de région par région. Certaines régions ont su bien profité. D'un côté Bretagne qui a su lier le dynamique économique qui est devenu une région forte. Et puis à côté la région centre qui existe à peine.
En Espagne, la Riogia se développe très vite et très bien, très vite que les régions françaises étudiées mais la Galice s'enfonce dans le sous-développement, reste périphérique, reste agricole et n'a pas réussi à se servir des financements européen pour se servir [...].
Le contexte Espagnol est plus favorable au contexte français, mais la capacité politique est meilleur au contexte français donc la Bretagne a su aussi se développer.

Est-ce que l'européanisation produit de la convergence? Non.

\section{Conclusion}

Les débats précédents en termes de jeux à somme nulle, c'est dépassé, pas toujours le même perdant ou gagnant, un jeu plus complexe avec plus d'acteurs que part le passé et qui finalement conforte le caractère irrésistible de la construction européenne puisque les acteurs régionaux y avait de l'intérêt aussi.
